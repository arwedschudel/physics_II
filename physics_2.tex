%%This is a very basic article template.
%%There is just one section and two subsections.
\documentclass{article}

%definitions for Formelsammlung

\usepackage[left=1.5cm,right=1.5cm,top=2.5cm,bottom=2cm,landscape]{geometry} 
\usepackage{multicol}
\usepackage[ngerman]{babel}
\usepackage{tabularx}
\usepackage{mathpazo}
\usepackage{mathtools}
\usepackage{amsmath}  
\usepackage{setspace} 
\usepackage{commath}
\usepackage[utf8]{inputenc}
%\usepackage[ansinew]{inputenc}  
\usepackage[T1]{fontenc}
\usepackage{lmodern} 
\usepackage{hyperref}
\usepackage{bigints}
\usepackage{array}
\usepackage[table]{xcolor}
\usepackage{layouts}
\usepackage{siunitx}
\usepackage{wrapfig}
\usepackage{multirow,bigstrut}
\usepackage{trfsigns}
\usepackage{amssymb} 
\usepackage{fancyhdr}
\usepackage{datetime}
\usepackage{pgfplots}
\usepgfplotslibrary{fillbetween}
\usepackage{listings}
\usepackage{mathrsfs}
\usepackage{tabu}
\usepackage{pdflscape}
\usepackage{booktabs}
%\usepackage{mathabx}
\usepackage{graphicx}
\usepackage{supertabular}
\usepackage{siunitx}
\usepackage[europeanvoltages, europeancurrents, europeanresistors, americaninductors, smartlabels]{circuitikz}

\DeclareMathOperator\arctanh{arctanh}
\DeclareMathOperator\arsinh{arsinh} 
\DeclareMathOperator\arcosh{arcosh}
\DeclareMathOperator\artanh{artanh}
\DeclareMathOperator\arcoth{arcoth} 
\DeclareMathOperator\sinc{sinc} 
\DeclareMathOperator\sgn{sgn} 
\DeclareMathOperator\LPF{LPF} 
\DeclareMathOperator\Q{Q} 
\DeclareMathOperator\erf{erf} 
\DeclareMathOperator\var{Var} 
\DeclareMathOperator\Cov{Cov} 
\DeclareMathOperator\floor{floor} 
\DeclareMathOperator\E{E} 
\DeclareMathOperator\NDFT{DFT} 
\DeclareMathOperator\IDFT{IDFT} 


%colorCodes
\definecolor{listinggray}{gray}{0.9}
\definecolor{lbcolor}{rgb}{0.95,0.95,0.95}
\definecolor{lightGray}{gray}{0.1}

\definecolor{cOrange}{HTML}{996633}
\definecolor{clOrange}{HTML}{DBB48D}
\definecolor{cBlue}{HTML}{336699}
\definecolor{clBlue}{HTML}{A0BCD8}
\definecolor{cGreen}{HTML}{339966}
\definecolor{clGreen}{HTML}{94D4B4}
\definecolor{cRed}{HTML}{993333}
\definecolor{clRed}{HTML}{D0B0B0}
\definecolor{cGray}{gray}{0.4}
\definecolor{clGray}{gray}{0.96}



\setlength{\parindent}{0pt}
%\DeclareMathOperator\arctanh{arccot}
\newcolumntype{L}[1]{>{\raggedright\let\newline\\\arraybackslash\hspace{0pt}}m{#1}}
\newcolumntype{C}[1]{>{\centering\let\newline\\\arraybackslash\hspace{0pt}}m{#1}}
\newcolumntype{R}[1]{>{\raggedleft\let\newline\\\arraybackslash\hspace{0pt}}m{#1}}
\newcolumntype{Y}{>{\centering\arraybackslash}X}
\newcolumntype{Z}{>{\raggedleft\arraybackslash}X}
\newcommand{\fmm}{\displaystyle} 
\newcommand{\cn}[1]{\underline{#1}} 
\newcommand{\hlaplace}{\quad\laplace\quad}
\newcommand{\hLaplace}{\quad\Laplace\quad}
\newcommand{\ztransform}{\, \, \xrightarrow{\, \, z\, \, } \, \,}
\newcommand{\zTransform}{\, \, \xrightarrow{\, \, z^{-1}\, \, } \, \, }
\newcommand{\infint}{\int_{-\infty}^{+\infty}}
\newcommand{\infiint}{\iint_{-\infty}^{+\infty}}
\newcommand{\limint}{\lim_{T\rightarrow \infty} \frac{1}{T} \int_{-T/2}^{T/2}}
\newcommand{\bedeq}{\mathrel{\stackrel{\makebox[0pt]{\mbox{\normalfont\tiny WSS}}}{=\joinrel=}}}
\renewcommand{\fourier}{\mathcal{F}}
\newcommand{\infsum}[1]{\sum_{#1 = -\infty}^{\infty} }
\newcommand{\cif}{\text{if}\:}
\newcommand{\cand}{\:\text{and}\:}
\newcommand{\celse}{\text{otherwise}\:}
\renewcommand{\abs}[1]{\left| #1 \right|}
\newcommand{\cvec}[1]{\left[\begin{smallmatrix} #1 \end{smallmatrix}\right]}
\newcommand{\vvec}[1]{\renewcommand*{\arraystretch}{0.8}\left[\begin{array}{c} #1 \end{array}\right]}
\renewcommand{\hat}[1]{\widehat{#1}}
\let\oldsi\si
\renewcommand{\si}[1]{\; \left[\oldsi[per-mode = fraction]{#1}\right]}

\newcommand{\plotTF}[1]{
\begin{tikzpicture}
\begin{axis}[xlabel=$\omega$,ylabel=$\abs{H(\omega}$, xmin = 0, xmax = 3.5, ymin = 0, ymax = 1, xtick = {3.14}, xticklabel={$\pi$}, ytick={0}, axis lines=middle, width=6cm, height=4cm, 
every axis x label/.style={at={(ticklabel* cs:1.05)},anchor=west}]
\addplot[name path=H, domain=0:3.14, samples=200] {#1};
\path[name path=axis] (axis cs:0,0.01) -- (axis cs:3.14,0.01);
\addplot[fill=clGray] fill between[of=H and axis, soft clip={domain=0.01:3.14}];
\end{axis}
\end{tikzpicture}
}

\newenvironment{donotbrake}{\begin{minipage}{\columnwidth}}{
\end{minipage} \vspace{1em}}

\newenvironment{cmat}[1]{
  \renewcommand*{\arraystretch}{0.9}
  \left[
  \begin{array}{#1}
}{
  \end{array}
  \right]
}

\newenvironment{case}{
  \left\{ \begin{array}{ll}
}{
  \end{array} \right.
}


\newenvironment{scase}{
  \renewcommand*{\arraystretch}{1}
  \left\{ \begin{array}{ll}
  }{
  \end{array} \right.
}

\newcommand\xdownarrow[1][2ex]{%
   \mathrel{\rotatebox{90}{$\xleftarrow{\rule{#1}{0pt}}$}}
}

\newcommand{\ncr}[2]{\binom{#1}{#2}}

\renewenvironment{description}{\color{cGray}}{}
\newenvironment{definition}{\color{cGray}}{} 
\newcommand{\cdef}[1]{\begin{definition}#1\end{definition}}


\newcommand{\vLaplace}[1][]{\mbox{\setlength{\unitlength}{0.1em}%
        \begin{picture}(10,20)%
          \put(3,2){\circle{4}}%
          \put(3,4){\line(0,1){12}}%
          \put(3,18){\circle*{4}}%
          \put(10,7){#1}
        \end{picture}%
       }%
 }%

\newcommand{\vlaplace}[1][]{\mbox{\setlength{\unitlength}{0.1em}%
        \begin{picture}(10,20)%
          \put(3,2){\circle*{4}}%
          \put(3,4){\line(0,1){12}}%
          \put(3,18){\circle{4}}%
          \put(10,7){#1}
        \end{picture}%
       }%
 }%         
 
           
\newenvironment{blockdiagram}[1]{
	\begin{tikzpicture}
		[auto, node distance=2.5cm,>=latex', scale=#1, every node/.style={scale=#1}]
}{
	\end{tikzpicture}
} 
 
 
\renewcommand{\arraystretch}{1.5}


\newenvironment{mtabular}[1] {
  \renewcommand{\arraystretch}{2}
  
  \begin{tabular}{#1}
}  
{
  \end{tabular}
  
  \renewcommand{\arraystretch}{1.5}
}

\newenvironment{lmtabular}[1] {
\renewcommand{\arraystretch}{2}

\begin{supertabular}{#1}
}  
{
\end{supertabular}

\renewcommand{\arraystretch}{1.5}
}

\newenvironment{dtabular} {
  \begin{tabular}{>{\begin{definition}}l<{\end{definition}} >{\begin{definition}}l<{\end{definition}}}
}  
{
  \end{tabular}
}

\newenvironment{ddtabular} {
  \begin{center}
  \begin{tabular}{>{\begin{definition}}l<{\end{definition}} >{\begin{definition}}l<{\end{definition}} | >{\begin{definition}}l<{\end{definition}} >{\begin{definition}}l<{\end{definition}}}
}  
{
  \end{tabular}
  \end{center}
}


%configure tikz
%system description
\usetikzlibrary{shapes,arrows}
\usetikzlibrary{decorations.markings}
\usetikzlibrary{calc}
\tikzstyle{block} = [draw, rectangle, minimum height=2.5em, minimum width=5em]
\tikzstyle{input} = [coordinate]
\tikzstyle{output} = [coordinate]
\tikzstyle{pinstyle} = [pin edge={to-,thin,black}]
\tikzstyle{sum} = [draw, circle, node distance=1em, minimum height=1.5em]
\tikzset{
	>=latex,
	photon/.style={decorate,decoration={snake,post length=1mm, segment length = 2mm, amplitude=0.6mm}}
}
\tikzset{->-/.style={decoration={
			markings,
			mark=at position #1 with {\arrow{>}}},postaction={decorate}}}
\tikzset{%
  block/.style    = {draw, thick, rectangle, minimum height = 2.5em,
    minimum width = 3em},
  sum/.style      = {draw, circle, node distance = 1.5cm}, % Adder
  input/.style    = {coordinate}, % Input
  output/.style   = {coordinate}, % Output
  gain/.style     = {draw, thick, isosceles triangle, minimum height = 2em, isosceles triangle apex angle=60},
  rgain/.style    = {draw, thick, isosceles triangle, minimum height = 2em, isosceles triangle apex angle=60}
}


%externalize TIKZ
%\usetikzlibrary{external}
%\tikzexternalize[prefix=tikz/]

%lstlisting

\lstset{
  backgroundcolor=\color{lbcolor},
  tabsize=2,    
% rulecolor=,
  language=[GNU]C++,
  basicstyle=\scriptsize,
  upquote=true,
  aboveskip={1.5\baselineskip},
  columns=fixed,
  showstringspaces=false,
  extendedchars=false,
  breaklines=true,
  prebreak = \raisebox{0ex}[0ex][0ex]{\ensuremath{\hookleftarrow}},
  frame=single,
  numbers=none,
  showtabs=false,
  showspaces=false,
  showstringspaces=false,
  identifierstyle=\ttfamily,
  keywordstyle=\color{cBlue}
  commentstyle=\color{cGreen},
  stringstyle=\color{cRed},
  numberstyle=\color{black},
% \lstdefinestyle{C++}{language=C++,style=numbers}’.
}
\lstset{
  backgroundcolor=\color{lbcolor},
  tabsize=2,
  language=C++,
  captionpos=b,
  tabsize=3,
  frame=lines,
  numbers=none,
  numberstyle=\tiny,
  numbersep=5pt,
  breaklines=true,
  showstringspaces=false,
  basicstyle=\ttfamily,
  identifierstyle=\color{cOrange},
  keywordstyle=\color{cBlue},
  commentstyle=\color{cGreen},
  stringstyle=\color{cRed}
}

\lstdefinelanguage{makefile}{
  morekeywords={cc,CFLAGS,LFLAGS,OBJ,EXE},
  morecomment=[l]{\#}
}

\lstdefinestyle{makefile}{
  language=makefile,
  basicstyle=\ttfamily,
  keywordstyle=\color{cBlue},
  commentstyle=\color{cGreen},
  frame=lines,
  numbers=none,
  backgroundcolor=\color{lbcolor}
}


\newcolumntype{M}{>{$}c<{$}} % math-mode version of "c" column type

%header & footer
\pagestyle{fancy}
\lhead{Tibor Schneider}
\rhead{Seite \thepage}
\cfoot{\today} 

\renewcommand{\headrulewidth}{0.4pt}
\renewcommand{\footrulewidth}{0.4pt}

%Title of Document
\chead{Physics II - Summary} 

\begin{document}
\begin{twocolumn} 



\section{The Photon}  

\begin{donotbrake}
\subsection{constants}
\begin{tabular}{cc}
	\begin{dtabular}
		
		$c \si{\metre \per \second}$ & speed of light \\
		$h \si{\metre \squared \kilogram \per \second}$ & planc's constant \\
		$e \si{\coulomb}$ & electorn charge \\
		$m_e \si{\kilogram}$ & electron mass \\
		$k_B \si{\metre \squared \kilogram \per \second \squared \per \kelvin}$ & bolzmann constant \\
		$\lambda \si{\metre}$ & Wavelength \\
		$\nu \si{\per \second}$ & Frequency \\
		$\omega \si{\radian \per \second}$ & Radial frequency \\
		$E \si{\joule}$ & Energy \\
		
	\end{dtabular}

	\begin{mtabular}{c}
		$c = 2.998 \cdot 10^8 \si{\metre \per \second}$ \\
		$h = 6.626 \cdot 10^{-34} \si{\metre \squared \kilogram \per \second}$ \\
		$\hslash = \frac{h}{2\pi}$ \\
		$e = 1.602 \cdot 10^{-19} \si{\coulomb}$ \\
		$m_e = 9.109 \cdot 10^{-31} \si{\kilogram}$ \\
		$k_B = 1.381 \cdot 10^{-23} \si{\metre \squared \kilogram \per \second \squared \per \kelvin}$ \\
		$1 \si{\electronvolt} = 1.602 \cdot 10^{-19} \si{\joule}$ \\
		$\fmm \lambda = \frac{c}{\nu} \qquad \fmm \nu = \frac{c}{\lambda} \qquad \omega = 2 \pi \nu$ \\
		$E = h \cdot \nu$ \\
	\end{mtabular}
\end{tabular}
\end{donotbrake}

\begin{donotbrake}
\subsection{Photoelectric effect}

\begin{tabular}{cc}
	\begin{dtabular}
		$V \si{\volt}$ & Voltage \\
		$\phi_0 \si{\electronvolt} $ & Work function \\
		$I \si{\ampere}$ & Photo-current \\
		$n \si{\metre^{-3}}$ & Volume density of electrons \\
		$A \si{\metre \squared}$ & Area \\
		$v \si{\metre \per \second}$ & velocity of electrons \\	
	\end{dtabular} &
	\begin{mtabular}{c}
		$\fmm h \nu - \phi_0 = \frac{1}{2} m v^2 = eV$ \\
		$\fmm V(\nu) =  \frac{h}{e} \nu - \frac{\phi_0}{e}$ \\
		$\fmm I = n A v e$ \\
	\end{mtabular}
\end{tabular}
\end{donotbrake}


\begin{donotbrake}
\subsection{Blackbody Radiation}

\begin{ddtabular}
	$L \si{\metre}$ & length of blackbody cube &
	$k_i$ & wave constants \\
	$E_x$ & Electric field in x-direction &
	$<E>$ & Average Energy \\
	$N$ & Number of states &
	$D$ & Density of states \\
	$u$ & Blackbody radiation &
	$I$ & Power radiated \\
\end{ddtabular}

\begin{mtabular}{c}
	$E_x(x,y,z) = E_{0x} \cos(k_x x) sin(k_y y) sin(k_z z)$ \\
	$\fmm k_x = n \frac{\pi}{L} \quad \fmm k_y = m \frac{\pi}{L} \quad \fmm k_z = l \frac{\pi}{L} \qquad k = \sqrt{k_x^2 + k_y^2 + k_z^2}$ \\
	$\fmm N(k) = \frac{1}{3\pi^2} k^3 L^3 \qquad D(k) = \frac{k^2}{\pi^2}$ \\
	$\fmm u(\omega) =\frac{\omega^2}{\pi^2 c^3} \cdot \frac{\hslash \omega}{\exp\left(\frac{-\hslash \omega}{k T}\right)-1} d\omega \qquad u(\nu) = \frac{8\pi h \nu^3}{c^3 \left( \exp \left(\frac{h \nu}{k T}\right) - 1\right)} d\nu$ \\
	$\fmm I(\omega) = c \cdot u(\omega)$
\end{mtabular}

\textbf{Equipartition-Theorem}: Each degree of Freedom has an energy of $kT$
\end{donotbrake}

\begin{donotbrake}
\subsection{Johnson-Noise}

This is the noise created in a one-dimensional circuit (like a coax-cable).

\begin{tabular}{cc}
	
	\begin{tabular}{c}
	  \begin{circuitikz} [scale=0.7, transform shape]
	  	\draw [thick] (1,2) to [short] (5,2);
	  	\draw [thick] (1,0) to [short] (5,0);
	  	\draw (1,0) to [short, o-] (0,0) to [R=$R$] (0,2) to [short, -o ](1,2);
	  	\draw (5,0) to [short, o-] (6,0) to [R=$R$] (6,2) to [short, -o ](5,2);
	  	\draw (3,1.2) node {$Z_c = R$};
	  	\draw [latex-latex] (1,0.3) -- (5,0.3) node [pos=0.5, above] {$L$};
  	\end{circuitikz} \\ $\qquad$
	  \begin{circuitikz} [scale=0.7, transform shape]
	  	\draw (0,0) to [R=$R'$] (0,2) to [sV_=$V$] (4,2) to [R=$R$] (4,0) to [short] (0,0);
	  	\draw [dashed] (-0.8,-0.6) rectangle (2.6,2.6);
	  \end{circuitikz}
	\end{tabular} &

	\begin{tabular}{c}
		\begin{dtabular}
			$\langle V^2\rangle$ & Noise Voltage \\
			$\Delta \nu$ & Bandwidth \\
		\end{dtabular} \\
		\begin{mtabular}{c}
			$E = E_0 \cdot \sin(k_x \cdot x)$ \\
			$\langle V^2 \rangle = 4 R \cdot k_BT \cdot \Delta \nu$
		\end{mtabular}
	\end{tabular}
\end{tabular}
\end{donotbrake}

\begin{donotbrake}
\subsection{Momentum of a photon}

\begin{tabular}{cc}
	\begin{tabular}{c}
		\begin{tikzpicture}
			\draw (2,0) rectangle (2.2,3);
			\draw [->, photon] (0,0.5) -- (2,1.5);
			\draw [->, photon] (1.9,1.6) -- (0,2.5);
			\draw [->] (2.2,1.5) -- (3,1.5) node[pos=0.5, above] {$v$};
		\end{tikzpicture}
	\end{tabular} &
	\begin{tabular}{c}
		\begin{dtabular}
			$p$ &momentum \\
		\end{dtabular} \\
		\begin{mtabular}{c}
			$\fmm p_{absorbing} =\frac{h \nu}{c}$ \\
			$\fmm p_{reflecting} =2 \cdot \frac{h \nu}{c}$
		\end{mtabular}
	\end{tabular}
\end{tabular}
\end{donotbrake}

\begin{donotbrake}
\subsection{Absorption, spontaneous and stimulated emission}
\begin{center}
\begin{tabular}{ccc}
	\begin{tikzpicture}
		\draw[thick] (0,0) -- (1,0) (0,2) -- (1,2);
		\draw[*->] (0.5,0) -- (0.5,2) node[pos=0.5, right]{$B_{12}$};
		\draw[->,photon] (-0.5,1) -- (0.5,1);
	\end{tikzpicture} &
	\begin{tikzpicture}
	\draw[thick] (0,0) -- (1,0) (0,2) -- (1,2);
	\draw[<-*] (0.5,0) -- (0.5,2) node[pos=0.5, left]{$A_{21}$};
	\draw[->,photon] (0.5,1) -- (1.5,1);
	\end{tikzpicture} &
	\begin{tikzpicture}
	\draw[thick] (0,0) -- (1,0) (0,2) -- (1,2);
	\draw[<-*] (0.5,0) -- (0.5,2) node[pos=0.8, right]{$B_{21}$};
	\draw[->,photon] (-0.5,1) -- (0.5,1);
	\draw[->,photon] (0.5,1.2) -- (1.5,1.2);
	\draw[->,photon] (0.5,0.8) -- (1.5,0.8);
	\end{tikzpicture} \\
	absorbtion & spontaneous emission & stimulated emission
\end{tabular}
\end{center}

\begin{center}
	\begin{dtabular}
		$n_1$ & Number of electrons in the lower energy state \\
		$n_2$ & Number of electrons in the higher energy state \\
	\end{dtabular} 
\end{center}

$$\fmm \frac{dn_2}{dt} = \underbrace{n_1 \cdot u(\nu) \cdot B_{12}}_\text{absorbtion} - \underbrace{n_2 \cdot u(\nu) \cdot B_{21}}_\text{stimulated emission} - \underbrace{n_2 \cdot A_{21}}_\text{spontaneous emission} $$
$$\fmm \frac{n_2}{n_1} = e^{-\frac{h \nu}{k_B T}} = \frac{u(\nu) B_{12}}{u(\nu) B_{21} + A_{21}}$$
$$\fmm B_{21} = B_{12} = B \qquad A_{21} = \frac{8\pi h \nu^3}{c^3}$$

\end{donotbrake}

\begin{donotbrake}
\subsection{Laser-optical amplification}

\begin{center}
	\begin{tabular}{cc}
		\begin{tikzpicture}
		\draw [-implies, double equal sign distance] (0.3,0.5) -- (0.3,3.5) node[pos=0.5, left] {R};
		\draw [->, dashed] (0.7,3.5) -- (1.5,3);
		\draw [->, dashed] (1.5,1) -- (0.7,0.5);
		\draw [->] (1.5,3) -- (1.5,1);
		\draw [thick] (0,0.5) -- (1,0.5) node[right]{0} (0,3.5) -- (1,3.5) node[right] {3} (1,3) -- (2,3) node[right] {2} (1,1) -- (2,1) node[right] {1};
		\end{tikzpicture} &
		\begin{tikzpicture}
		\draw [fill=black!5] (1,0) -- (1,2) -- (5,2) node[pos=0.5, above] {amplification medium} -- (5,0) -- (1,0);
		\draw [thick] (0,0) arc (-150:-210:2);
		\draw [thick] (6,0) arc (-30:30:2);
		\draw [->-=.35] (-0.1,1.8) arc ( 250: 290:9.05);
		\draw [->-=.75] (-0.1,1.8) arc ( 250: 290:9.05);
		\draw [->-=.75] ( 6.1,1.8) arc ( -70: -110:9.05);
		\draw [->-=.35] ( 6.1,1.8) arc ( -70: -110:9.05);
		\draw [->-=.35] (-0.1,0.2) arc (-250:-290:9.05);
		\draw [->-=.75] (-0.1,0.2) arc (-250:-290:9.05);
		\draw [->-=.75] ( 6.1,0.2) arc (70:110:9.05);
		\draw [->-=.35] ( 6.1,0.2) arc (70:110:9.05);
		\draw [implies-,double equal sign distance] (3,0) -- (3,-0.5) node[below] {pump};
		\draw [->, dashed] (6.3,1) -- (7.3,1);
		\draw [->, dashed] (6.2,1.5) -- (7.3,1.5);
		\draw [->, dashed] (6.2,0.5) -- (7.3,0.5);
		\draw (6,2) node[above] {cavity};
		\end{tikzpicture}
	\end{tabular}
\end{center}

Electrons are excited from the ground state ``0'' to the level ``3'' by pumping through incoherent radiation. 
The electrons then fall onto a long-lived state $n_2$ (State ``2'') from level ``3''. 
The pumping can be done either optically by shining a strong incoherent light or by passing a current. 
It is also assumed that the lower state is quickly emptied by a fast process with lifetime $\tau_1$. 
As a result, the population in state ``2'' is:
$$n_2 = \frac{R}{A_{21}} \quad \text{whereas} \quad n_1 \approx 0 \quad \text{because} \quad  A_{21} < \frac{1}{\tau_1}$$
We have rherefore a population inversion between the two states. 
The likelihood of a stimulated emission process is larger than the one of absorbtion. 
If we enclose the system in an optical cavity, we can achieve self-sustained oscillation at the frequency $\nu$.

\end{donotbrake}

\section{Wave mechanics}

\begin{center}
	\begin{tabular}{ccccc}
		& frequency & wavelength & momentum & energy \\
		\midrule
		Particle & & $\lambda_b = \frac{h}{p}$ & $p = m v$ & $E = \frac{1}{2} m v^2$ \\
		Wave & $\omega$ & $\lambda = \frac{2\pi c}{\omega}$ & $p = \frac{\hslash \omega}{c}$ & $E = \hslash \omega$ \\
	\end{tabular}
\end{center}

\begin{donotbrake}
\subsection{Compton Scattering}

\begin{tabular}{cc}
	\begin{tabular}{c}
		\begin{tikzpicture}
			\draw [->, photon] (0,0) -- (1.3,0) node[pos=0.5, below] {$p_1$};
			\draw [fill=black] (2,-0.1) circle (0.08) node[below] {$e^-$};
			\draw [->, photon] (4,0) -- (5.5,1) node[pos=0.5, above left]{$p_2$};
			\draw [fill=black] (4,0) circle (0.08) node[below] {$e^-$};
			\draw [->] (4,0) -- (5.5,-1.3) node[pos=0.7, below] {$v$};
			\draw [dashed] (4,0) -- (5.5,0);
			\draw (4.8,0) arc (0:33:0.8) node[pos=0.6, right] {$\theta$};
			\draw (4.9,0) arc (0:-41:0.9) node[pos=0.6, right] {$\phi$};
		\end{tikzpicture}
	\end{tabular} &
	\begin{mtabular}{c}
		$\fmm p_1 =\frac{h \nu_1}{c} \qquad p' = \frac{h \nu_2}{c}$ \\
		$\fmm \nu_2 = \nu_1 - \frac{P_e^2}{2 m_e h}$ \\
		$\fmm \lambda_2 - \lambda_1 = \frac{h}{m_e c} (1 - \cos \theta)$;
	\end{mtabular}
\end{tabular}
\end{donotbrake}


\begin{donotbrake}
	\subsection{Bragg diffraction}
	
	\begin{tabular}{cc}
		\begin{tabular}{c}
			\begin{tikzpicture}
			\draw [thick] (0,0.05) -- (0,0.35);
			\foreach \y in {0.4,0.8,1.2,1.6,2.0} {
				\draw [thick] (-0.05,\y - 0.05) -- (0.05,\y - 0.05);
				\draw [thick] (-0.05,\y + 0.05) -- (0.05,\y + 0.05);
				\draw [thick] (0,\y + 0.05) -- (0,\y + 0.35);
			}
			\draw (0.2,0.6) node {\small $a$};
			\draw [dashed] (0,1.6) -- (2,1.6);
			\draw (0,1.6) -- (2,1);
			\draw (1.5,1.6) arc (0:-17:1.5);
			\draw (1.2,1.4) node {\small $\theta$}; 
			\end{tikzpicture}
		\end{tabular} &
		\begin{mtabular}{c}
			$\fmm \sin \theta = \frac{n \lambda}{a}$
		\end{mtabular}
	\end{tabular}
\end{donotbrake}

\begin{donotbrake}
\subsection{Single slit}

\begin{tabular}{cc}
	\begin{tabular}{c}
		\begin{tikzpicture}
			\draw [->, photon] (-1.5,1) -- (-0.7,1);
			\draw [thick] (0,-1) -- (0,0.7) -- (0.1,0.7) -- (0.1,-1);
			\draw [draw=none, fill=black!10] (0,-1) rectangle (0.1,0.7); 
			\draw [thick] (0,3) -- (0,1.3) -- (0.1,1.3) -- (0.1,3);
			\draw [draw=none, fill=black!10] (0,3) rectangle (0.1,1.3);
			\draw [|-|] (-0.2,0.7) -- (-0.2,1.3) node[pos=0.5, left] {$a$};
			\draw (2,-1) -- (2,3);
			\draw [thick, domain=-2.01:2, samples=150, variable=\y] plot ({30*sin(\y*200)*sin(\y*200)/((\y*20)*(\y*20)) + 2},{\y + 1}); 
			\draw (2.5,1.8) node {$I(\theta)$};
			\draw [dashed] (0,1) -- (2,1);
			\draw (0,1) -- (1.95,1.75);
			\draw (1.5,1) arc (0:21:1.5) node[pos=0.5, left] {$\theta$};
		\end{tikzpicture}
	\end{tabular} &
	\begin{mtabular}{c}
		$\fmm I(\theta) = I_0 \frac{\sin^2 \theta}{\theta^2}$ \\
		$\fmm \sin \theta = \frac{\lambda}{a}$
	\end{mtabular}
\end{tabular}
\end{donotbrake}

\end{twocolumn}

\end{document}
