%definitions for Formelsammlung

\usepackage[left=1.5cm,right=1.5cm,top=2.5cm,bottom=2cm,landscape]{geometry} 
\usepackage{multicol}
\usepackage[ngerman]{babel}
\usepackage{tabularx}
\usepackage{mathpazo}
\usepackage{mathtools}
\usepackage{amsmath}  
\usepackage{setspace} 
\usepackage{commath}
\usepackage[utf8]{inputenc}
%\usepackage[ansinew]{inputenc}  
\usepackage[T1]{fontenc}
\usepackage{lmodern} 
\usepackage{hyperref}
\usepackage{bigints}
\usepackage{array}
\usepackage[table]{xcolor}
\usepackage{layouts}
\usepackage{siunitx}
\usepackage{wrapfig}
\usepackage{multirow,bigstrut}
\usepackage{trfsigns}
\usepackage{amssymb} 
\usepackage{fancyhdr}
\usepackage{datetime}
\usepackage{pgfplots}
\usepgfplotslibrary{fillbetween}
\usepackage{listings}
\usepackage{mathrsfs}
\usepackage{tabu}
\usepackage{pdflscape}
\usepackage{booktabs}
%\usepackage{mathabx}
\usepackage{graphicx}
\usepackage{supertabular}
\usepackage{siunitx}
\usepackage[europeanvoltages, europeancurrents, europeanresistors, americaninductors, smartlabels]{circuitikz}
\usepackage{xparse}

\DeclareMathOperator\arctanh{arctanh}
\DeclareMathOperator\arsinh{arsinh} 
\DeclareMathOperator\arcosh{arcosh}
\DeclareMathOperator\artanh{artanh}
\DeclareMathOperator\arcoth{arcoth} 
\DeclareMathOperator\sinc{sinc} 
\DeclareMathOperator\sgn{sgn} 
\DeclareMathOperator\LPF{LPF} 
\DeclareMathOperator\Q{Q} 
\DeclareMathOperator\erf{erf} 
\DeclareMathOperator\var{Var} 
\DeclareMathOperator\Cov{Cov} 
\DeclareMathOperator\floor{floor} 
\DeclareMathOperator\E{E} 
\DeclareMathOperator\NDFT{DFT} 
\DeclareMathOperator\IDFT{IDFT} 


%colorCodes
\definecolor{listinggray}{gray}{0.9}
\definecolor{lbcolor}{rgb}{0.95,0.95,0.95}
\definecolor{lightGray}{gray}{0.1}

\definecolor{cOrange}{HTML}{996633}
\definecolor{clOrange}{HTML}{DBB48D}
\definecolor{cBlue}{HTML}{336699}
\definecolor{clBlue}{HTML}{A0BCD8}
\definecolor{cGreen}{HTML}{339966}
\definecolor{clGreen}{HTML}{94D4B4}
\definecolor{cRed}{HTML}{993333}
\definecolor{clRed}{HTML}{D0B0B0}
\definecolor{cGray}{gray}{0.4}
\definecolor{clGray}{gray}{0.96}



\setlength{\parindent}{0pt}
%\DeclareMathOperator\arctanh{arccot}
\newcolumntype{L}[1]{>{\raggedright\let\newline\\\arraybackslash\hspace{0pt}}m{#1}}
\newcolumntype{C}[1]{>{\centering\let\newline\\\arraybackslash\hspace{0pt}}m{#1}}
\newcolumntype{R}[1]{>{\raggedleft\let\newline\\\arraybackslash\hspace{0pt}}m{#1}}
\newcolumntype{Y}{>{\centering\arraybackslash}X}
\newcolumntype{Z}{>{\raggedleft\arraybackslash}X}
\newcommand{\fmm}{\displaystyle} 
\newcommand{\cn}[1]{\underline{#1}} 
\newcommand{\hlaplace}{\quad\laplace\quad}
\newcommand{\hLaplace}{\quad\Laplace\quad}
\newcommand{\ztransform}{\, \, \xrightarrow{\, \, z\, \, } \, \,}
\newcommand{\zTransform}{\, \, \xrightarrow{\, \, z^{-1}\, \, } \, \, }
\newcommand{\infint}{\int_{-\infty}^{+\infty}}
\newcommand{\infiint}{\iint_{-\infty}^{+\infty}}
\newcommand{\limint}{\lim_{T\rightarrow \infty} \frac{1}{T} \int_{-T/2}^{T/2}}
\newcommand{\bedeq}{\mathrel{\stackrel{\makebox[0pt]{\mbox{\normalfont\tiny WSS}}}{=\joinrel=}}}
\renewcommand{\fourier}{\mathcal{F}}
\newcommand{\infsum}[1]{\sum_{#1 = -\infty}^{\infty} }
\newcommand{\cif}{\text{if}\:}
\newcommand{\cand}{\:\text{and}\:}
\newcommand{\celse}{\text{otherwise}\:}
\renewcommand{\abs}[1]{\left| #1 \right|}
\newcommand{\cvec}[1]{\left[\begin{smallmatrix} #1 \end{smallmatrix}\right]}
\newcommand{\vvec}[1]{\renewcommand*{\arraystretch}{0.8}\left[\begin{array}{c} #1 \end{array}\right]}
\renewcommand{\hat}[1]{\widehat{#1}}
\let\oldsi\si
\renewcommand{\si}[1]{\; \left[\oldsi[per-mode = fraction]{#1}\right]}

\ExplSyntaxOn
\clist_new:N \l_feq_vector_clist
\NewDocumentCommand{\Vector}{O{\\}mO{b}}{
	\clist_set:Nn \l_feq_vector_clist {#2} % Set the list
	\renewcommand*{\arraystretch}{1}
	\begin{#3matrix}
		\clist_use:Nn \l_feq_vector_clist {#1} % show it with separator from #1 (\\)
	\end{#3matrix}
}
\ExplSyntaxOff

\newcommand{\plotTF}[1]{
\begin{tikzpicture}
\begin{axis}[xlabel=$\omega$,ylabel=$\abs{H(\omega}$, xmin = 0, xmax = 3.5, ymin = 0, ymax = 1, xtick = {3.14}, xticklabel={$\pi$}, ytick={0}, axis lines=middle, width=6cm, height=4cm, 
every axis x label/.style={at={(ticklabel* cs:1.05)},anchor=west}]
\addplot[name path=H, domain=0:3.14, samples=200] {#1};
\path[name path=axis] (axis cs:0,0.01) -- (axis cs:3.14,0.01);
\addplot[fill=clGray] fill between[of=H and axis, soft clip={domain=0.01:3.14}];
\end{axis}
\end{tikzpicture}
}

\newenvironment{donotbrake}{\begin{minipage}{\columnwidth}}{
\end{minipage} \vspace{1em}}

\newenvironment{cmat}[1]{
  \renewcommand*{\arraystretch}{0.9}
  \left[
  \begin{array}{#1}
}{
  \end{array}
  \right]
}

\newenvironment{case}{
  \left\{ \begin{array}{ll}
}{
  \end{array} \right.
}


\newenvironment{scase}{
  \renewcommand*{\arraystretch}{1}
  \left\{ \begin{array}{ll}
  }{
  \end{array} \right.
}

\newcommand\xdownarrow[1][2ex]{%
   \mathrel{\rotatebox{90}{$\xleftarrow{\rule{#1}{0pt}}$}}
}

\newcommand{\ncr}[2]{\binom{#1}{#2}}

\renewenvironment{description}{\color{cGray}}{}
\newenvironment{definition}{\color{cGray}}{} 
\newcommand{\cdef}[1]{\begin{definition}#1\end{definition}}


\newcommand{\vLaplace}[1][]{\mbox{\setlength{\unitlength}{0.1em}%
        \begin{picture}(10,20)%
          \put(3,2){\circle{4}}%
          \put(3,4){\line(0,1){12}}%
          \put(3,18){\circle*{4}}%
          \put(10,7){#1}
        \end{picture}%
       }%
 }%

\newcommand{\vlaplace}[1][]{\mbox{\setlength{\unitlength}{0.1em}%
        \begin{picture}(10,20)%
          \put(3,2){\circle*{4}}%
          \put(3,4){\line(0,1){12}}%
          \put(3,18){\circle{4}}%
          \put(10,7){#1}
        \end{picture}%
       }%
 }%         
 
           
\newenvironment{blockdiagram}[1]{
	\begin{tikzpicture}
		[auto, node distance=2.5cm,>=latex', scale=#1, every node/.style={scale=#1}]
}{
	\end{tikzpicture}
} 
 
 
\renewcommand{\arraystretch}{1.5}


\newenvironment{mtabular}[1] {
  \renewcommand{\arraystretch}{2}
  
  \begin{tabular}{#1}
}  
{
  \end{tabular}
  
  \renewcommand{\arraystretch}{1.5}
}

\newenvironment{lmtabular}[1] {
\renewcommand{\arraystretch}{2}

\begin{supertabular}{#1}
}  
{
\end{supertabular}

\renewcommand{\arraystretch}{1.5}
}

\newenvironment{dtabular} {
  \begin{tabular}{>{\begin{definition}}l<{\end{definition}} >{\begin{definition}}l<{\end{definition}}}
}  
{
  \end{tabular}
}

\newenvironment{ddtabular} {
  \begin{center}
  \begin{tabular}{>{\begin{definition}}l<{\end{definition}} >{\begin{definition}}l<{\end{definition}} | >{\begin{definition}}l<{\end{definition}} >{\begin{definition}}l<{\end{definition}}}
}  
{
  \end{tabular}
  \end{center}
}


%configure tikz
%system description
\usetikzlibrary{shapes,arrows}
\usetikzlibrary{decorations.markings}
\usetikzlibrary{calc}
\tikzstyle{block} = [draw, rectangle, minimum height=2.5em, minimum width=5em]
\tikzstyle{input} = [coordinate]
\tikzstyle{output} = [coordinate]
\tikzstyle{pinstyle} = [pin edge={to-,thin,black}]
\tikzstyle{sum} = [draw, circle, node distance=1em, minimum height=1.5em]
\tikzset{
	>=latex,
	photon/.style={decorate,decoration={snake,post length=1mm, segment length = 2mm, amplitude=0.6mm}}
}
\tikzset{->-/.style={decoration={
			markings,
			mark=at position #1 with {\arrow{>}}},postaction={decorate}}}
\tikzset{%
  block/.style    = {draw, thick, rectangle, minimum height = 2.5em,
    minimum width = 3em},
  sum/.style      = {draw, circle, node distance = 1.5cm}, % Adder
  input/.style    = {coordinate}, % Input
  output/.style   = {coordinate}, % Output
  gain/.style     = {draw, thick, isosceles triangle, minimum height = 2em, isosceles triangle apex angle=60},
  rgain/.style    = {draw, thick, isosceles triangle, minimum height = 2em, isosceles triangle apex angle=60}
}


%externalize TIKZ
%\usetikzlibrary{external}
%\tikzexternalize[prefix=tikz/]

%lstlisting

\lstset{
  backgroundcolor=\color{lbcolor},
  tabsize=2,    
% rulecolor=,
  language=[GNU]C++,
  basicstyle=\scriptsize,
  upquote=true,
  aboveskip={1.5\baselineskip},
  columns=fixed,
  showstringspaces=false,
  extendedchars=false,
  breaklines=true,
  prebreak = \raisebox{0ex}[0ex][0ex]{\ensuremath{\hookleftarrow}},
  frame=single,
  numbers=none,
  showtabs=false,
  showspaces=false,
  showstringspaces=false,
  identifierstyle=\ttfamily,
  keywordstyle=\color{cBlue}
  commentstyle=\color{cGreen},
  stringstyle=\color{cRed},
  numberstyle=\color{black},
% \lstdefinestyle{C++}{language=C++,style=numbers}’.
}
\lstset{
  backgroundcolor=\color{lbcolor},
  tabsize=2,
  language=C++,
  captionpos=b,
  tabsize=3,
  frame=lines,
  numbers=none,
  numberstyle=\tiny,
  numbersep=5pt,
  breaklines=true,
  showstringspaces=false,
  basicstyle=\ttfamily,
  identifierstyle=\color{cOrange},
  keywordstyle=\color{cBlue},
  commentstyle=\color{cGreen},
  stringstyle=\color{cRed}
}

\lstdefinelanguage{makefile}{
  morekeywords={cc,CFLAGS,LFLAGS,OBJ,EXE},
  morecomment=[l]{\#}
}

\lstdefinestyle{makefile}{
  language=makefile,
  basicstyle=\ttfamily,
  keywordstyle=\color{cBlue},
  commentstyle=\color{cGreen},
  frame=lines,
  numbers=none,
  backgroundcolor=\color{lbcolor}
}


\newcolumntype{M}{>{$}c<{$}} % math-mode version of "c" column type

%header & footer
\pagestyle{fancy}
\lhead{Tibor Schneider}
\rhead{Seite \thepage}
\cfoot{\today} 

\renewcommand{\headrulewidth}{0.4pt}
\renewcommand{\footrulewidth}{0.4pt}
